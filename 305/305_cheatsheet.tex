%%%%%%%%%%%%%%%%%%%%%%%%%%%%%%%%%%%%%%%%%%%%%%%%%%%%%%%%%%%%%%%%%%%%%%
% writeLaTeX Example: A quick guide to LaTeX
%
% Source: Dave Richeson (divisbyzero.com), Dickinson College
% 
% A one-size-fits-all LaTeX cheat sheet. Kept to two pages, so it 
% can be printed (double-sided) on one piece of paper
% 
% Feel free to distribute this example, but please keep the referral
% to divisbyzero.com
% 
%%%%%%%%%%%%%%%%%%%%%%%%%%%%%%%%%%%%%%%%%%%%%%%%%%%%%%%%%%%%%%%%%%%%%%
% How to use writeLaTeX: 
%
% You edit the source code here on the left, and the preview on the
% right shows you the result within a few seconds.
%
% Bookmark this page and share the URL with your co-authors. They can
% edit at the same time!
%
% You can upload figures, bibliographies, custom classes and
% styles using the files menu.
%
% If you're new to LaTeX, the wikibook is a great place to start:
% http://en.wikibooks.org/wiki/LaTeX
%
%%%%%%%%%%%%%%%%%%%%%%%%%%%%%%%%%%%%%%%%%%%%%%%%%%%%%%%%%%%%%%%%%%%%%%

\documentclass[10pt,landscape]{article}
\usepackage{amssymb,amsmath,amsthm,amsfonts}
\usepackage{multicol,multirow}
\usepackage{calc}
\usepackage{ifthen}
\usepackage[landscape]{geometry}
\usepackage[colorlinks=true,citecolor=blue,linkcolor=blue]{hyperref}


\ifthenelse{\lengthtest { \paperwidth = 11in}}
    { \geometry{top=.5in,left=.5in,right=.5in,bottom=.5in} }
	{\ifthenelse{ \lengthtest{ \paperwidth = 297mm}}
		{\geometry{top=1cm,left=1cm,right=1cm,bottom=1cm} }
		{\geometry{top=1cm,left=1cm,right=1cm,bottom=1cm} }
	}
\pagestyle{empty}
\makeatletter
\renewcommand{\section}{\@startsection{section}{1}{0mm}%
                                {-1ex plus -.5ex minus -.2ex}%
                                {0.5ex plus .2ex}%x
                                {\normalfont\large\bfseries}}
\renewcommand{\subsection}{\@startsection{subsection}{2}{0mm}%
                                {-1explus -.5ex minus -.2ex}%
                                {0.5ex plus .2ex}%
                                {\normalfont\normalsize\bfseries}}
\renewcommand{\subsubsection}{\@startsection{subsubsection}{3}{0mm}%
                                {-1ex plus -.5ex minus -.2ex}%
                                {1ex plus .2ex}%
                                {\normalfont\small\bfseries}}
                                
\newcommand\todo[1]{\textcolor{red}{#1}}

\makeatother
\setcounter{secnumdepth}{0}
\setlength{\parindent}{0pt}
\setlength{\parskip}{0pt plus 0.5ex}

\theoremstyle{definition}
\newtheorem*{question}{Question}
\newtheorem{theorem}{Theorem}
\newtheorem{defin}{Definition}
\newtheorem{type}{Type}
\theoremstyle{remark}
\newtheorem*{remark}{Remark}


% -----------------------------------------------------------------------

\title{MATH 305 Cheatsheet}

\begin{document}

\raggedright
\footnotesize

\begin{center}
     \Large{\textbf{MATH 305 Cheatsheet with \LaTeX}} \\
\end{center}
\begin{multicols}{3}
\setlength{\premulticols}{1pt}
\setlength{\postmulticols}{1pt}
\setlength{\multicolsep}{1pt}
\setlength{\columnsep}{2pt}

\section{Summary}
\subsection{Basic}
1. Fundamentals of complex variable. Euler’s formula. Polar coordinate. Principal value of argument Arg (z)\\
2. Arg (z) and arg (z). De Moivre’s formula. Roots of unit. Roots of a
complex variable \\
3. Complex exponential. Sets in the complex plane. Functions of complex
variables \\
4. Functions of complex variables. Image under linear and Mobius map $w = \frac{a+bz}{c+dz}$\\
5. Image under $w = z^2$. Continuous, differentiable, analytic. Cauchy-Riemann equation\\
6. Consequences of Cauchy-Riemann equation. Harmonic Functions. Conformal Mapping. Level Sets\\
7. Laplace under analytical mappings. ∂z¯f (z) = 0. Conformal Mappings. \\
8. Elementary functions $e^z$ and $sin z$. Images under $e^z$ and $sin z$ \\
9. Properties of sin z and sinh z. Introduction of Log (z)\\
10. Multi-valued functions. introduction of log z and Log (z) and their properties \\
11. Multi-valued functions. Introduction of $z^\alpha$ and branch cuts \\
12. Multi-valued functions. Branch cuts for $(z^2-1)^\frac{1}{2}$ \\
13. Branch cuts for $(z^3-z)^{\frac{1}{2}}$, $(z^3-z)^{\frac{1}{3}}$, $(z^2+1)^{\frac{1}{2}}$ \\
14. Inverse function of sin z. Solving Laplace equation with Arg (z) \\
\subsection{Complex Integral Basics}
15. Complex integrals. Contours (Paths) \\
16. Fundamental Theorem of Calculus in the Complex Case. \\
17. Cauchy-Coursat Theorem. simply-connected domains. Path independence and deformation of path \\
18. Path Independence. Cauchy Integral Formula. \\
19. Applications of Cauchy Integral Formula. Computation of real integrals \\
20. Consequences of Cauchy Integral Formula. Functions with finite order 
singularity \\
21. Consequences of Cauchy Integral Formula. Liouville Theorem: bounded 
entire functions are constants\\
22. Maximum Modulus Principle \\
23. Argument Principle, Nyquist Criterion \\
24. Argument Principle, Nyquist criterion, applications to ODE \\
25. Rouche’s Theorem. Classification of Singularities\\
26. Classification of singularities and computations of residues\\
27. Cauchy Residue Theorem. Computation of residues and contour integrals \\
\subsection{Application}
28. Applications of Cauchy Residue Theorem\\
29. Type I, Type II real integrals\\
30. Type III real integrals\\
31. Type IV and Type V integrals. Integrals involving Multi-valued functions \\
32. Type V integrals \\
33. Fourier transforms and inverse Fourier transforms. Two Properties. 
Applications to ODE and PDE\\

%%%%%%%%%%%%%%%%%%%%%%%%%%%%%%%%%%%%%%%%%%%%%%%%%%%%%%%%%%%
\section{Analytic}
for f(z) = u + iv, if:
\begin{itemize}
    \item exist $u_x,u_y,v_x,v_y$ in D
    \item $u_x,u_y,v_x_v_y$ are continuous in D
    \item satisfy C-R equation in D
\end{itemize}
Then f(z) is analytic in D

%%%%%%%%%%%%%%%%%%%%%%%%%%%%%%%%%%%%%%%%%%%%%%%%%%%%%%%%%%%

\section{Harmonic}
let $\Delta \phi = \partial^2_x \phi +  \partial^2_y \phi = 0$ for $\phi(x, y)$ and f(z) = u + iv, if f(z) is analytic, then u, v are harmonic.

\textbf{Given function and its domain, find it's image.} \\
Basic mapping example, given region and boundary, ask for harmonic function: 
\begin{itemize}
    \item washer: $Log|z|$
    \item lines: Arg(z)
\end{itemize}
Solve it by observe that $f(z) = \alpha z + \beta$. 

\begin{question}
Find harmonic function u(x,y) in the right half plane, with the boundary condition of u(0, y) = 1 on $y \in (1, \infty)$, u(0, y) = 2 in $y \in (-\infty, 1)$
\end{question}
\section{Branch Cut}
\begin{itemize}
    \item break down the function with $e^log(f(z))$
    \item separate argument part of the function.
    \item choose branch cut for each section to meet the requirement domain.
    \item check the initial value, if not meet, shift the whole branch cut by $2\pi$.
    \item note that the correction could be in arugment or just shifting.
\end{itemize}

\begin{question}
find a branch cut for $f(z) = (z(z+2)(z-3))^\frac{1}{2}$, so that it's analytic in C \ {$(-\infty,-2] \cup [0, 3]$} and f(-1) = 2.
\end{question}

\section{Image}
\subsection{Cauchy-Riemann Equation}
$$u_x = v_y, u_y = -u_x$$
\begin{question}
given imaginary part of f(z), and some initial value, find f(z)
\end{question}

\subsection{Multivalued function}
$$sin^{-1}z = -iLog(iZ + \sqrt{1-z^2})$$


\subsection{Confrontal Mapping}
Given function, Mapping one domain to another. \\
Find the transformation function to map given domain to another. 
\begin{itemize}
    \item shifting: $w = z + c$
    \item rotation: $w = e^{i\theta} z$
    \item mobius: $w = \frac{az+b}{cz+d}$, mapping from line or circle to itself or to each other.
    \item $w = sin(z)$: strip to line. 
    \item $w = log(z)$: circle to strip.
\end{itemize}


%%%%%%%%%%%%%%%%%%%%%%%%%%%%%%%%%%%%%%%%%%%%%%%%%%%%%%%%%%%
\section{Nyquist Criterion}
Let N be the number of roots inside C, for P(z) are second order or higher taking the form $y^n + a_oy^(n-1)+ ... = 0$:
\begin{align*}
    N &= \frac{1}{2\pi}\int_C \frac{P'(z)}{P(z)}\\
    N & = \frac{1}{2\pi}[argP(z)]|_c \\
    N & = \frac{1}{2\pi}(n+\pi + 2[argP(z)]|_\Gamma_+) \\
\end{align*}
where $\Gamma_+$ is $0<y<\infty$, traverse downwards. \\ 
\todo{$[argP(z)]|_c$ denote the change in the argument of P(z) over C. }
\begin{itemize}
    \item Find the degree pf poly as n.
    \item Set up P(iy) and find the 
\end{itemize}


%%%%%%%%%%%%%%%%%%%%%%%%%%%%%%%%%%%%%%%%%%%%%%%%%%%%%%%%%%%
\section{Cauchy's theorem}


\subsection{Laurent Series}
keep this identity in mind: 
\color{red}
$$\frac{1}{1-z} = \sum^{\infty}_{j=0}z^j$$, if $|z|<1$
\color{black}

\begin{question}
find Laurent series expansion for $\frac{1}{(z-2)z}$ in $0<|z|<2$ and $2<|z| < \infty$
\end{question}


%%%%%%%%%%%%%%%%%%%%%%%%%%%%%%%%%%%%%%%%%%%%%%%%%%%%%%%%%%%

\section{Residue Theorem}
\begin{theorem}
if f has an isolated singularity at point $z_0$, then the coefficient of $a_{-1}$ of $\frac{1}{z-z_0}$ in the Laurent expansion for f around $z_0$ is called the residue of f at $z_0$, denoted by $Res(z_0)$, where $\int_\Gamma f(z)dz = 2\pi i \cdot Res(z_0)$
\end{theorem}
Given condition P(z), Q(z) are analytic at $z_0$ and Q has a simple zero at $z_0$, $P(z_0) \neq 0$, we can use $Res(z_0) = \frac{P(z_0)}{Q'(z_0)}$


\begin{defin}
if f has a pole of order m at $z_0$, then:
$$Res(z_0) = \lim_{z \rightarrow z_0} \frac{1}{(m-1)!}\frac{d^{m-1}}{dz^{m-1}}[(z-z_0)^mf(z)]$$
This applies in general, always check the degree of the poles!
\end{defin}

\begin{theorem}
\textbf{Cauchy's Residue Theorem}: if $\Gamma$ is a simple closed positively oriented contour, f is analytic inside and on $\Gamma$ except points $z_i$ inside $\Gamma$, then $\int_{\Gamma}f(z)dz = 2\pi i \sum^n_{j=1}Res(z_j)$ 
\end{theorem}
\todo{ALWAYS CHECK IF $z_i$ is inside $\Gamma$}

\begin{type} Trigonometric \par
Real integral of the form: $\int^{2\pi}_0 U(cos(\theta), sin(\theta))d\theta$. Replace $\theta$ with z get $\int_C U(\frac{1}{2}(z+\frac{1}{z}), \frac{1}{2i}(z-\frac{1}{z}))\cdot\frac{1}{iz} dz$
\end{type}

\begin{question}
$I = \int^\pi_0 \frac{d\theta}{2 - cos(\theta)}$
\end{question}

\begin{type} Improper \par
If $f(z) = \frac{P(z)}{Q(z)}$ such that degree Q $\geq$ 2 + degree P, then $$\lim_{p \rightarrow  \infty} \int_{C_p^+} f(z)dz = 0$$, where $C^+_p is the upper half-circle of radius p.$
\end{type}

\begin{question}
$I = \int^\infty_{-\infty} \frac{x^2}{(x^2 + 1)^2}$
\end{question}

\begin{type} improper Trigonometric \par
Real integrals of the form $I = \int^\infty_{-\infty} \frac{P(x)}{Q(x)}cos(mx) dx $ or $I = \int^\infty_{-\infty} \frac{P(x)}{Q(x)} sin(mx) dx or I = \int^\infty_{-\infty} \frac{P(x)}{Q(x)} e^{imx} dx $, where Q(x) are 2 degrees higher than P(x).
\end{type}
\begin{remark}
\todo{Q(x) and P(x) are both defined and continuous over the interval!}
\end{remark}

\begin{theorem} Jordan's Lemma \par
if n $>$ 0  and P/Q is the quotient of two polynomials such that degree Q $\geq$ 1 + degree P, then $$\lim_{p \rightarrow \infty} \int_{C^+_p} e^{imz} \frac{P(z)}{Q(z)} dz = 0$$, where $C^+_p$ is the upper half-circle of radius p. 
\end{theorem}

\begin{remark}
The proof of this theorem uses a limiting method to bound sin(t) by $\frac{2t}{\pi}$ between (0, $\frac{\pi}{2}$), then 
\end{remark}

\begin{question}
$I = \int^\infty_{-\infty} \frac{e^{ix}x}{x^2+1} dx$
\end{question}

\begin{remark}
Also, if there is only cos(x) or sin(x), think of it as the Re(f(z)) or Im(f(z))
\end{remark}

\todo{When doing half circle, be careful with the direction of the contour}


\begin{type} Indented contour \par
If f has a simple pole at z = c and $T_r$ is the circular arc defined by $T_r: z = c + re^{i\theta}, {\theta_1 \leq \theta \leq \theta2}$. Then $$\lim_{r \rightarrow 0^+} \int_{T_r} f(z) dz = i (\theta_2 -\theta_1) Res(f;c)$$, where positive orientation is counterclockwise. 
\end{type}
\begin{question}
$I = \int^\infty_{-\infty} \frac{e^{ix}x}{x^2-1} dx$
\end{question}

\begin{type} Multiple-Valued \par
When f(z) includes a multi-valued function, we break $ f(z ) \rightarrow e^{\alpha Log(x)}$ and create a $\gamma_1, \gamma_2$ contour on the branch cut, then create closed contour to calculate the two contour, based on $$\int_\Gamma f(z) dz = 2\pi i \cdot \sum(residue of f at poles inside \Gamma)$$  
\end{type}

\begin{question}
$I = \int^\infty_0 \frac{dx}{x^\lambda(x-4)}, here 0 < \lambda < 1$
\end{question}

\begin{remark}
\todo{Note that the inner circle has a different direction!}
\end{remark}

\section{Transformation}
\subsection{Use Fourier to solve ODE}
\begin{itemize}
    \item Take first order ODE and replace x with $i\omegat$ to get the homogeneous equation, and the equivalent g(t) using fourier transformation $g_f(w) = \frac{1}{2\pi}\int^\infty_{-\infty}g(t)e^{-i\omega t}dt$ 
    \item get $f(w)$ and transform back to $f(t)$ with $f(t) = \frac{1}{2\pi}\int^\infty_{-\infty}f_f(w)e^{-i\omega t}dw$
\end{itemize}



\section{Miscs}



\end{multicols}

\end{document}
